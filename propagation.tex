\section{Radio Wave Propagation}

\subsection{Electromagnetic Waves}

\subsection{Ionosphere}

The ionoshpere is a layer of Earth's atmosphere from roughly 60 - 1000 km in altitude. Solar radiation, primarily in the ultraviolet spectrum, ionizes the air molecules creating free electrons that affect radio wave propagation. Since it is dependent on the sun, the intensity of ionization in the ionosphere changes depending on the time of day. During midday, the ionosphere is most ionized, while shortly before dawn, it is least ionized.

The ionosphere is classified into multiple layers, depending on which ionized molecules are present and the effect this has on radio wave propagation. Figure~\ref{fig:ionosphere} shows the different layers and their respective presence during day and night.

\subsection{Skip}

\subsection{Fading}

\subsection{Solar Activity}

\subsection{Critical and Maximum Usable Frequency}

\subsection{Ducting and Sporadic-E}

\subsection{Scatter Propagation}

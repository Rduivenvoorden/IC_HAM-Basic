\documentclass[12pt,letterpaper,fleqn,oneside]{article}

% Packages
\usepackage{sectsty}
\usepackage{caption}
%\usepackage{pslatex}
%\usepackage{times}
\usepackage{amssymb,amsfonts,amsmath,amscd}
%\usepackage[pdftex,colorlinks]{hyperref}
\usepackage[pdftex]{graphicx}
\usepackage{fnpos}
\usepackage{subfigure}
\usepackage{verbatim}
\usepackage{fancyhdr}
\usepackage{pifont}
\usepackage{setspace}

\newcommand{\tbf}[1]{\textbf{#1}}

\newenvironment{ppl}{\fontfamily{ppl}\selectfont}{\par}

% Set the page size
\addtolength{\hoffset}{-0.75in} \addtolength{\voffset}{-0.75in}
\setlength{\textwidth}{7in} \setlength{\textheight}{8.25in}
\setlength{\headheight}{0in}
\setlength{\headsep}{0.4in}

\setlength{\footskip}{40pt}
\setlength{\fboxsep}{12pt}

% Set the paragraph skip
\setlength{\parskip}{12pt}
\setlength{\parindent}{0pt}

\begin{document}

\LARGE{\textbf{Basic Amateur Radio Qualification}}

\normalsize For obtaining the Industry Canada Amateur Radio Operator Certificate with Basic Qualification. Guide by Rduivenvoorden VA3PRD.

% started in preparation of the HAM Basic Exam on March 23, 2015 after finding minimal studying material. Intended to be available to all who are interested to learn, or are preparing for their basic HAM exam.

\include{regulations}

\include{procedures}

\include{station_assembly}

\include{circuit_assembly}

\include{basics}

\include{antennas}

\section{Radio Wave Propagation}

\subsection{Electromagnetic Waves}

There are several methods by which electromagnetic waves propagate through the Earth's atmosphere. At lower frequencies, the wavelength becomes very large and are affected by the Earth and the atmosphere.

\subsection{Ionosphere}

The ionoshpere is a layer of Earth's atmosphere from roughly 60 - 1000 km in altitude. Solar radiation, primarily in the ultraviolet spectrum, ionizes the air molecules creating free electrons that affect radio wave propagation. Since it is dependent on the sun, the intensity of ionization in the ionosphere changes depending on the time of day. During midday, the ionosphere is most ionized, while shortly before dawn, it is least ionized.

The ionosphere is classified into multiple layers, depending on which ionized molecules are present and the effect this has on radio wave propagation. Figure~\ref{fig:ionosphere} shows the different layers and their respective presence during day and night.

\subsection{Skip}

\subsection{Fading}

\subsection{Solar Activity}

\subsection{Critical and Maximum Usable Frequency}

\subsection{Ducting and Sporadic-E}

\subsection{Scatter Propagation}


\section{Interference Suppression}

\subsection{Front-End Overload and Cross Modulation}

\subsection{Audio Rectification}

\subsection{Spurious Emissions and Key Clicks}

\subsection{Harmonics and Splatter}

\subsection{Filters}


\end{document}

\documentclass[12pt,letterpaper,fleqn,oneside]{article}

% Packages
\usepackage{sectsty}
\usepackage{caption}
%\usepackage{pslatex}
%\usepackage{times}
\usepackage{amssymb,amsfonts,amsmath,amscd}
\usepackage[pdftex,colorlinks]{hyperref}
\usepackage[pdftex]{graphicx}
\usepackage{fnpos}
\usepackage{subfigure}
\usepackage{verbatim}
\usepackage{fancyhdr}
\usepackage{pifont}
\usepackage{setspace}

\newcommand{\tbf}[1]{\textbf{#1}}

\newenvironment{ppl}{\fontfamily{ppl}\selectfont}{\par}

% Set the page size
\addtolength{\hoffset}{-0.75in} \addtolength{\voffset}{-0.75in}
\setlength{\textwidth}{7in} \setlength{\textheight}{8.25in}
\setlength{\headheight}{0in}
\setlength{\headsep}{0.4in}

\setlength{\footskip}{40pt}
\setlength{\fboxsep}{12pt}

% Set the paragraph skip
\setlength{\parskip}{12pt}
\setlength{\parindent}{0pt}

\begin{document}

\LARGE{\textbf{Basic Amateur Radio Qualification}}

\normalsize For obtaining the Industry Canada Amateur Radio Operator Certificate with Basic Qualification. Guide by Rduivenvoorden VA3PRD.

% started in preparation of the HAM Basic Exam on March 23, 2015 after finding minimal studying material. Intended to be available to all who are interested to learn, or are preparing for their basic HAM exam.

\section{Regulations and Policies}

\section{Operating Procedures}

\section{Station Assembly}

\section{Circuit Components}

\section{Basic Electronics and Theory}

\section{Feedlines and Antenna Systems}

\section{Radio Wave Propagation}

\section{Interference Suppression}

\end{document}
